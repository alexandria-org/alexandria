
\documentclass[12pt, a4paper]{article}
\usepackage[T1]{fontenc}
\usepackage[utf8]{inputenc}
\usepackage[swedish]{babel}

\title{Stadgar för Föreningen Alexandria.org}
\date{Januari 2022}

\begin{document}

\maketitle

\paragraph{§ 1 Föreningens firma}
\paragraph{}
Föreningens firma är Föreningen Alexandria.org och föreningens firmatecknare är ordförande eller annan person utsedd till firmatecknare av styrelsen.

\paragraph{§ 2 Föreningens ändamål}
\paragraph{}
Föreningen har som ändamål att göra kunskap mer tillgängligt. Föreningen ska uppfylla sitt ändamål genom att utveckla och tillhandahålla en sökmotor som är gratis och utan annonser. Källkoden till sökmotorn ska publiceras som öppen källkod.

\paragraph{§ 3 Föreningens säte}
\paragraph{}
Föreningen har sitt säte i Uppsala.

\paragraph{§ 4 Medlemsskap}
\paragraph{}
Föreningens medlemmar är aktiva i föreningens verksamhet. Nya medlemmar måste godkännas av styrelsen.

\paragraph{§ 5 Medlemsavgifter}
\paragraph{}
Medlem ska betala den medlemsavgift som årligen fastställs av årsmötet.

\pagebreak


\paragraph{§ 6 Styrelsen}
\paragraph{}
Styrelsen består av en ordförande, en kassör, en suppleant och eventuellt ytterligare ledarmöter enligt årsmötets beslut.

\paragraph{§ 7 Styrelsens uppdrag}
\paragraph{}
Styrelsen företräder föreningen, bevakar dess intressen och handhar dess angelägenheter. Styrelsen beslutar å föreningens vägnar såvida inte annat
föreskrivs i dessa stadgar. Styrelsen ska verkställa av årsmötet fattade beslut, handha föreningens ekonomiska angelägenheter och föra räkenskaper,
samt avge årsredovisning till årsstämman för det senaste räkenskapsåret. Styrelsen sammanträder när ordföranden finner det erforderligt eller om
minst två styrelseledamöter begär detta.

\paragraph{}
Styrelsen är beslutsför då ordförande och minst en ledamot är närvarande. Styrelsebeslut fattas med enkel majoritet. Vid lika röstetal gäller den mening
ordföranden biträder.

\paragraph{§ 8 Räkenskaper}
\paragraph{}
Räkenskapsår ska vara kalenderår.

\paragraph{§ 9 Revisor}
\paragraph{}
Styrelsens förvaltning ska årligen granskas av en på årsmötet utsedd revisor. Revisorn ska senast den 1 mars avge sin revisionsberättelse. Revisorn får ej vara medlem i styrelsen.

\paragraph{§ 10 Årsmöte}
\paragraph{}
Ordinarie årsmöte, vilket är föreningens högsta beslutande organ, hålls årligen före den 30 juni på tid och plats som styrelsen bestämmer. Kallelse sker via epost minst 1 vecka före utsatt möte.

\paragraph{}
Vid ordinarie årsmöte ska följande ärenden behandlas:
\begin{enumerate}
\item Val av ordförande och sekreterare för mötet.
\item Fastställande av röstlängd för mötet.
\item Fastställande av dagordning.
\item Styrelsens verksamhetsberättelse för det senaste verksamhetsåret.
\item Styrelsens förvaltningsberättelse (balans- och resultaträkning) för det senaste verksamhets-/räkenskapsåret.
\item Revisionsberättelsen för verksamhets-/räkenskapsåret.
\item Fråga om ansvarsfrihet för styrelsen för den tid revisionen avser.
\item Fastställande av medlemsavgifter.
\item Fastställande av ev. verksamhetsplan och behandling av budget för det kommande verksamhets-/räkenskapsåret.
\item Val av ordförande i föreningen för en tid av 1 år.
\item Val av kassör, övriga styrelseledamöter samt suppleanter för en tid av 1 år
\item Val av revisorer.
\item Behandling av styrelsens förslag och i rätt tid inkomna motioner.
\item Övriga frågor. 
\end{enumerate}

\paragraph{§ 11 Extra årsmöte}
\paragraph{}
Extra årsmöte hålls när styrelsen eller revisorerna finner att det är nödvändigt. Kallelse sker via epost minst 1 vecka före utsatt möte.

\paragraph{§ 12 Rösträtt}
\paragraph{}
Vid årsmöte har varje medlem en röst. Rösträtten är personlig och kan inte utövas genom ombud.

\paragraph{§ 13 Beslut, omröstning och beslutsmässighet}
\paragraph{}
Beslut fattas med bifallsrop (acklamation) eller om så begärs, efter omröstning (votering).

\paragraph{}
Omröstning sker öppet, utom vid val där sluten omröstning ska äga rum om någon begär detta. Beslut fattas, såvida dessa stadgar ej föreskriver
annat, med enkel majoritet. Vid lika röstetal har ordförande rätt att bestämma.

\paragraph{}
Mötet är beslutsmässigt med det antal röstberättigade medlemmar som är närvarande på mötet.

\paragraph{§ 14 Regler för ändring av stadgarna}
\paragraph{}
För ändring av dessa stadgar krävs beslut av två på varandra följande årsmöten. Förslag till ändring av stadgarna får ges såväl av medlem som styrelsen.

\paragraph{§ 15 Utträde}
\paragraph{}
Medlem som önskar utträda ur föreningen ska skriftligen anmäla detta till styrelsen och anses därmed omedelbart ha lämnat föreningen.

\paragraph{§ 16 Uteslutning}
\paragraph{}
Medlem får uteslutas från föreningen om den har försummat att betala beslutade avgifter, motarbetat föreningens
verksamhet eller ändamål, eller skadat föreningens intressen. Beslut om uteslutning fattas av styrelsen.

\end{document}
